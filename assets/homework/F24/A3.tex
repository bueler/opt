\documentclass[12pt]{amsart}
%prepared in AMSLaTeX, under LaTeX2e
\addtolength{\oddsidemargin}{-.6in} 
\addtolength{\evensidemargin}{-.6in}
\addtolength{\topmargin}{-.4in}
\addtolength{\textwidth}{1.2in}
\addtolength{\textheight}{.6in}

\renewcommand{\baselinestretch}{1.05}

\usepackage{verbatim,fancyvrb}

\usepackage{palatino}

\newtheorem*{thm}{Theorem}
\newtheorem*{defn}{Definition}
\newtheorem*{example}{Example}
\newtheorem*{problem}{Problem}
\newtheorem*{remark}{Remark}

\newcommand{\mtt}{\texttt}
\usepackage{alltt,xspace}
\newcommand{\mfile}[1]
{\medskip\begin{quote}\scriptsize \begin{alltt}\input{#1.m}\end{alltt} \normalsize\end{quote}\medskip}

\usepackage[final]{graphicx}
\newcommand{\mfigure}[1]{\includegraphics[height=2.5in,
width=3.5in]{#1.eps}}
\newcommand{\regfigure}[2]{\includegraphics[height=#2in,
keepaspectratio=true]{#1.eps}}
\newcommand{\widefigure}[3]{\includegraphics[height=#2in,
width=#3in]{#1.eps}}

\usepackage{amssymb}

\usepackage[pdftex, colorlinks=true, plainpages=false, linkcolor=black, citecolor=red, urlcolor=red]{hyperref}

% macros
\newcommand{\br}{\mathbf{r}}
\newcommand{\bv}{\mathbf{v}}
\newcommand{\bx}{\mathbf{x}}
\newcommand{\by}{\mathbf{y}}

\newcommand{\CC}{\mathbb{C}}
\newcommand{\RR}{\mathbb{R}}
\newcommand{\ZZ}{\mathbb{Z}}

\newcommand{\eps}{\epsilon}
\newcommand{\grad}{\nabla}
\newcommand{\lam}{\lambda}
\newcommand{\lap}{\triangle}

\newcommand{\ip}[2]{\ensuremath{\left<#1,#2\right>}}

%\renewcommand{\det}{\operatorname{det}}
\newcommand{\onull}{\operatorname{null}}
\newcommand{\rank}{\operatorname{rank}}
\newcommand{\range}{\operatorname{range}}

\newcommand{\Julia}{\textsc{Julia}\xspace}
\newcommand{\Matlab}{\textsc{Matlab}\xspace}
\newcommand{\Octave}{\textsc{Octave}\xspace}
\newcommand{\Python}{\textsc{Python}\xspace}

\newcommand{\prob}[1]{\bigskip\noindent\textbf{#1}\quad }

\newcommand{\chapexers}[2]{\prob{Chapter #1, pages #2, Exercises:}}
\newcommand{\exer}[2]{\prob{Exercise #1}}

\newcommand{\pts}[1]{(\emph{#1 pts}) }
\newcommand{\epart}[1]{\medskip\noindent\textbf{(#1)}\quad }
\newcommand{\ppart}[1]{\,\textbf{(#1)}\quad }

\newcommand*\circled[1]{\tikz[baseline=(char.base)]{
            \node[shape=ellipse,draw,inner sep=2pt] (char) {#1};}}


\begin{document}
\scriptsize \noindent Math 661 Optimization (Bueler) \hfill 7 September 2024 (version 1)
\normalsize

\medskip\bigskip

\Large\centerline{\textbf{Assignment \#3}}
\large
\bigskip

\centerline{\textbf{Due Monday, 23 September 2024, at the start of class}}
\bigskip
\normalsize

\thispagestyle{empty}

\bigskip
\noindent From the textbook

\begin{quote}
Griva, Nash, and Sofer, \emph{Linear and Nonlinear Optimization}, 2nd ed., SIAM Press 2009.
\end{quote}

\noindent please read sections 2.4 and 2.6, and then 3.1 and 3.2.  Also read Appendices B.1 through B.8, noting you should have already read B.4 and B.5.  We will get back to sections 2.5 and 2.7 when they are in suitable context.

\bigskip
\noindent \textsc{Do the following exercise} from section 2.4, page 58:

\begin{itemize}
\item Exercise 4.2
\item Exercise 4.3
\end{itemize}

\bigskip
\noindent \textsc{Do the following exercises} from section 2.6, page 66:

\begin{itemize}
\item Exercise 6.3 (\emph{``First three terms'' means the terms which are constant, linear, and quadratic in $p$.  Do not go to 3rd order or 3rd derivatives.})
\item Exercise 6.4 (\emph{Ditto.})
\item Exercise 6.5
\end{itemize}

\bigskip
\noindent \textsc{Do the following exercises} from section 3.1, page 82:

\begin{itemize}
\item Exercise 1.2
\item Exercise 1.3
\end{itemize}

\end{document}
