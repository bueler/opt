\documentclass[12pt]{amsart}
%prepared in AMSLaTeX, under LaTeX2e
\addtolength{\oddsidemargin}{-.6in} 
\addtolength{\evensidemargin}{-.6in}
\addtolength{\topmargin}{-.4in}
\addtolength{\textwidth}{1.2in}
\addtolength{\textheight}{.6in}

\renewcommand{\baselinestretch}{1.05}

\usepackage{verbatim,fancyvrb}

\usepackage{palatino}

\newtheorem*{thm}{Theorem}
\newtheorem*{defn}{Definition}
\newtheorem*{example}{Example}
\newtheorem*{problem}{Problem}
\newtheorem*{remark}{Remark}

\newcommand{\mtt}{\texttt}
\usepackage{alltt,xspace}
\newcommand{\mfile}[1]
{\medskip\begin{quote}\scriptsize \begin{alltt}\input{#1.m}\end{alltt} \normalsize\end{quote}\medskip}

\usepackage[final]{graphicx}
\newcommand{\mfigure}[1]{\includegraphics[height=2.5in,
width=3.5in]{#1.eps}}
\newcommand{\regfigure}[2]{\includegraphics[height=#2in,
keepaspectratio=true]{#1.eps}}
\newcommand{\widefigure}[3]{\includegraphics[height=#2in,
width=#3in]{#1.eps}}

\usepackage{amssymb}

\usepackage[pdftex, colorlinks=true, plainpages=false, linkcolor=black, citecolor=red, urlcolor=red]{hyperref}

% macros
\newcommand{\br}{\mathbf{r}}
\newcommand{\bv}{\mathbf{v}}
\newcommand{\bx}{\mathbf{x}}
\newcommand{\by}{\mathbf{y}}

\newcommand{\CC}{\mathbb{C}}
\newcommand{\RR}{\mathbb{R}}
\newcommand{\ZZ}{\mathbb{Z}}

\newcommand{\eps}{\epsilon}
\newcommand{\grad}{\nabla}
\newcommand{\lam}{\lambda}
\newcommand{\lap}{\triangle}

\newcommand{\ip}[2]{\ensuremath{\left<#1,#2\right>}}

%\renewcommand{\det}{\operatorname{det}}
\newcommand{\onull}{\operatorname{null}}
\newcommand{\rank}{\operatorname{rank}}
\newcommand{\range}{\operatorname{range}}

\newcommand{\Julia}{\textsc{Julia}\xspace}
\newcommand{\Matlab}{\textsc{Matlab}\xspace}
\newcommand{\Octave}{\textsc{Octave}\xspace}
\newcommand{\Python}{\textsc{Python}\xspace}

\newcommand{\prob}[1]{\bigskip\noindent\textbf{#1}\quad }

\newcommand{\chapexers}[2]{\prob{Chapter #1, pages #2, Exercises:}}
\newcommand{\exer}[2]{\prob{Exercise #1}}

\newcommand{\pts}[1]{(\emph{#1 pts}) }
\newcommand{\epart}[1]{\medskip\noindent\textbf{(#1)}\quad }
\newcommand{\ppart}[1]{\,\textbf{(#1)}\quad }

\newcommand*\circled[1]{\tikz[baseline=(char.base)]{
            \node[shape=ellipse,draw,inner sep=2pt] (char) {#1};}}


\begin{document}
\scriptsize \noindent Math 661 Optimization (Bueler) \hfill 7 September, 2022
\normalsize

\medskip\bigskip

\Large\centerline{\textbf{Assignment \#2}}
\large
\bigskip

\centerline{\textbf{Due Monday, 19 September 2022, at the start of class}}
\bigskip
\normalsize

\thispagestyle{empty}

\bigskip
\noindent Make sure you have a copy of the textbook:

\begin{quote}
Griva, Nash, and Sofer, \emph{Linear and Nonlinear Optimization}, 2nd ed., SIAM Press 2009.
\end{quote}

\noindent Please read sections 2.1 through 2.4 and Appendices B.4 through B.8.

\bigskip
\noindent \textsc{Do the following exercise} from section 2.2, page 48:

\begin{itemize}
\item Exercise 2.7
\end{itemize}

\bigskip
\noindent \textsc{Do the following exercises} from section 2.3, pages 52--54:

\begin{itemize}
\item Exercise 3.1
\item Exercise 3.3
\item Exercise 3.7
\item Exercise 3.13
\item Exercise 3.18
%\item Exercise 3.19  \quad (\emph{only do parts} (iii)--(vii))
\item Exercise 3.20
\end{itemize}

\prob{Problem P5.}  For each of the following functions, determine if it is convex, concave, both, or neither, on the real line $\RR$.  (Explain your answer.)  If the function is convex or concave, indicated whether that is also strict.

\epart{a}  $f(x) = 8x - 15$

\epart{b}  $f(x) = \sqrt{1 + x^2}$

\epart{c}  $f(x) = 1 / (2+x^4)$

\epart{d}  $f(x) = |x|$

\epart{e}  $f(x) = 4 - 5 x - 3 x^2$


\prob{Problem P6.}  (\emph{This problem is related to Appendices B.4, B.6, B.7.})  Consider the scalar-valued function
    $$f(x_1,x_2) = \exp(-x_1^2 + 3 x_1 - 2 x_2 - x_2^2)$$
Compute the gradient and Hessian of $f$.  Find the location where $f$ is maximum, and explain what properties of the gradient and Hessian show that it is a maximum.

\clearpage \newpage
\prob{Problem P7.}  (\emph{This problem is related to Appendix B.4.  Carefully read about the relationship between the gradient of a vector-valued function and its Jacobian.})

\epart{a}  Consider the vector-valued function
    $$f(x_1,x_2,x_3) = \begin{pmatrix}  x_1^2 + x_2^2 + x_3^2 - 1 \\
                                        x_2 - \arctan(x_1) \\
                                        x_3^3 - x_3 - x_2 \end{pmatrix}$$
Compute the Jacobian of $f$.

\epart{b}  The Jacobian of $f$ in part \textbf{(a)} is a $3 \times 3$ matrix, just like the Hessian of a scalar function.  Is the Jacobian of $f$ in part \textbf{(a)} actually the Hessian of some scalar function?  This question can be answered by considering the symmetry of a matrix, for example at the point $x=(1,1,1)$ for concreteness.

\end{document}
