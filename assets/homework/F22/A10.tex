\documentclass[12pt]{amsart}
%prepared in AMSLaTeX, under LaTeX2e
\addtolength{\oddsidemargin}{-.7in} 
\addtolength{\evensidemargin}{-.7in}
\addtolength{\topmargin}{-.6in}
\addtolength{\textwidth}{1.4in}
\addtolength{\textheight}{1.1in}

\renewcommand{\baselinestretch}{1.05}

\usepackage{verbatim,fancyvrb}
\usepackage{soul}
\usepackage{palatino}
\usepackage{tikz}

\newtheorem*{thm}{Theorem}
\newtheorem*{defn}{Definition}
\newtheorem*{example}{Example}
\newtheorem*{problem}{Problem}
\newtheorem*{remark}{Remark}

\newcommand{\mtt}{\texttt}
\usepackage{alltt,xspace}
\newcommand{\mfile}[1]
{\medskip\begin{quote}\scriptsize \begin{alltt}\input{#1.m}\end{alltt} \normalsize\end{quote}\medskip}

\newcommand{\mfigure}[1]{\includegraphics[height=2.5in,
width=3.5in]{#1.eps}}
\newcommand{\regfigure}[2]{\includegraphics[height=#2in,
keepaspectratio=true]{#1.eps}}
\newcommand{\widefigure}[3]{\includegraphics[height=#2in,
width=#3in]{#1.eps}}

\usepackage{amssymb}

\usepackage[pdftex, colorlinks=true, plainpages=false, linkcolor=black, citecolor=red, urlcolor=red]{hyperref}

% macros
\newcommand{\br}{\mathbf{r}}
\newcommand{\bv}{\mathbf{v}}
\newcommand{\bx}{\mathbf{x}}
\newcommand{\by}{\mathbf{y}}

\newcommand{\CC}{\mathbb{C}}
\newcommand{\RR}{\mathbb{R}}
\newcommand{\ZZ}{\mathbb{Z}}

\newcommand{\eps}{\epsilon}
\newcommand{\grad}{\nabla}
\newcommand{\lam}{\lambda}
\newcommand{\lap}{\triangle}

\newcommand{\ip}[2]{\ensuremath{\left<#1,#2\right>}}

%\renewcommand{\det}{\operatorname{det}}
\newcommand{\onull}{\operatorname{null}}
\newcommand{\rank}{\operatorname{rank}}
\newcommand{\range}{\operatorname{range}}

\newcommand{\Julia}{\textsc{Julia}\xspace}
\newcommand{\Matlab}{\textsc{Matlab}\xspace}
\newcommand{\Octave}{\textsc{Octave}\xspace}
\newcommand{\Python}{\textsc{Python}\xspace}

\newcommand{\prob}[1]{\bigskip\noindent\textbf{#1}\quad }

\newcommand{\chapexers}[2]{\prob{Chapter #1, pages #2, Exercises:}}
\newcommand{\exer}[2]{\prob{Exercise #1}}

\newcommand{\pts}[1]{(\emph{#1 pts}) }
\newcommand{\epart}[1]{\medskip\noindent\textbf{(#1)}\quad }
\newcommand{\ppart}[1]{\,\textbf{(#1)}\quad }

\newcommand*\circled[1]{\tikz[baseline=(char.base)]{
            \node[shape=ellipse,draw,inner sep=2pt] (char) {#1};}}


\begin{document}
\scriptsize \noindent Math 661 Optimization (Bueler) \hfill 29 November, 2022
\normalsize

\medskip\bigskip

\Large\centerline{\textbf{Assignment \#10}}
\large
\bigskip

\centerline{\textbf{Due Friday, 9 December 2022, at the start of class}}
\bigskip
\normalsize

\thispagestyle{empty}

\bigskip
Please read sections 14.1--14.7 and 15.1--15.4 from the textbook.\footnote{Griva, Nash, and Sofer, \emph{Linear and Nonlinear Optimization}, 2nd ed., SIAM Press 2009.}

\bigskip
\noindent \textsc{Do the following exercises} from section 14.2, pages x--y:

\begin{itemize}
\item Exercise 2.7 \quad \begin{minipage}[t]{4.5in} \emph{Use techniques from either section 14.2 or 14.3.} \end{minipage}
\end{itemize}

\medskip
% F2 from F18 Final Exam
\prob{Problem P21.} Suppose $c\in\RR^n$ is a nonzero vector and consider the problem
\begin{alignat*}{2}
    \text{minimize}   &&  z = c^\top &x \\
    \text{subject to} && \quad \sum_{i=1}^n x_i^2 &= 1
\end{alignat*}
where $x\in\RR^n$.  Note there is a single equality constraint, which can be written $\|x\|^2=1$.

\epart{a}  By arguing informally explain why the solution is
    $$x_* = - \frac{c}{\|c\|}.$$
Use a sketch of the $n=2$ case to explain.

\epart{b}  The necessary optimality conditions for this problem are addressed by Theorem 14.15 on page 504 of the textbook.  Compute the Lagrangian and state the first-order necessary conditions in detail.  (\emph{Note that you do \emph{not} need to compute a null-space matrix in order to do this.})

\epart{c}  Solve the conditions in \textbf{(b)} algebraically to confirm the solution in part \textbf{(a)}.  How many points $(x_*,\lambda_*)$ are there which satisfy the first-order necessary conditions?


\medskip
% F3
\prob{Problem P22.}  (\emph{Before doing this problem read Example 14.20 on pages 506--507.  This problem asks for a similar analysis.})  Consider the problem
\begin{alignat*}{2}
    \text{minimize}   &&  f(x) &= (x_1-1)^2 + (x_2+1)^2 \\
    \text{subject to} && \quad x_1^2 + x_2^2 &\le 9 \\
                      &&        x_2 &\ge 0
\end{alignat*}

\epart{a}  Sketch the feasible set and explain informally, perhaps using contours of $f$, why $x_*=(1,0)^\top$ is the solution.

\epart{b}  Write the constraints in the form $g_i(x)\ge 0$.  Compute the Lagrangian and its gradient.  For each of the points $A = (0,0)^\top$, $B=(0,3)^\top$, and $C=(1,0)^\top$ compute the values of $\lambda_i$ satisfying the zero-gradient condition.  Address whether these points satisfy the first-order optimality conditions, that is, whether they are candidates for a local minimizer.  Show in particular that $C$ satisfies all the first-order conditions in Theorem 14.18.  (\emph{You do \emph{not} need to find any null-space matrices in order to answer this question.})


\medskip
% F5
\prob{Problem P23.}  Consider nonlinear optimization problems on $x\in \RR^n$ which have standard-form linear constraints:
    $$\begin{matrix}
    \text{minimize} \qquad & f(x) \\
    \text{subject to} \qquad & Ax = b \\
                      & x \ge 0
    \end{matrix}$$
Assume $f$ is smooth, that there are $m$ scalar constraint equations, and that $A$ has full row rank.  Thus $A\in \RR^{m\times n}$, $b\in \RR^m$, and $m\le n$ (as usual).

In 2D ($n=2$) there are exactly three possibilities for the dimension of the feasible set.  The cartoons below illustrate these possibilities when the feasible set $S$ is ``generic'', meaning non-empty and bounded (for $m>0$).

\bigskip
\begin{tikzpicture}[scale=0.75]
\filldraw [gray!50] (0,0) -- (4.9,0) -- (4.9,3.9) -- (0,3.9) -- cycle;
\draw [->] (-0.5,0)--(5,0) node[right] {$x_1$};
\draw [->] (0,-0.5)--(0,4) node[above] {$x_2$};
\node at (2,2) {\Large $S$};
\node at (2,-1) {$m=0$};
\end{tikzpicture}
\qquad
\begin{tikzpicture}[scale=0.75]
\draw [->] (-0.5,0)--(5,0) node[right] {$x_1$};
\draw [->] (0,-0.5)--(0,4) node[above] {$x_2$};
\draw [very thick] (0,3) -- (3.5,0);
\draw [->] (2.1,2.1) node[xshift=2mm,yshift=2mm] {\Large $S$} -- (1.7,1.7);
\node at (2,-1) {$m=1$};
\end{tikzpicture}
\qquad
\begin{tikzpicture}[scale=0.75]
\draw [->] (-0.5,0)--(5,0) node[right] {$x_1$};
\draw [->] (0,-0.5)--(0,4) node[above] {$x_2$};
\draw [black,fill=black] (1.5,1.5) circle (0.5mm);
\draw [->] (2.1,2.1) node[xshift=2mm,yshift=2mm] {\Large $S$} -- (1.7,1.7);
\node at (2,-1) {$m=2$};
\end{tikzpicture}

\bigskip
For 3D ($n=3$) there are four generic possibilities $m=0,1,2,3$.  Sketch the four corresponding cartoons.  These cartoons should have the same annotations as the 2D versions above.


\medskip
% F6
\prob{Problem P24.}  (\emph{This problem asks you to write the general KKT conditions.})  Consider the general nonlinear constrained optimization problem
\begin{alignat*}{2}
    \text{minimize}   &&  &f(x) \\
    \text{subject to} && \qquad g_i(x) &= 0, \quad i=1,\dots,\ell \\
                      &&       h_i(x) &\ge 0, \quad i=1,\dots,m
\end{alignat*}

\epart{a}  In section 14.5 the meaning of the phrase ``$x_*$ \emph{is a regular point of the constraints}'' is given.  State this definition precisely for the above problem.  Also state the Lagrangian for this problem.

\epart{b}  Suppose $x_*$ is a local minimizer of the above problem which is also a regular point.  State the first-order necessary conditions for the above problem.  (\emph{Again note that you do \emph{not} need to find any null-space matrices $Z(x)$ in order to answer this question.})


% consider using F4, which is a grid-search wrapper around an unconstrained method

\end{document}
