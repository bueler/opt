\documentclass[12pt]{amsart}
%prepared in AMSLaTeX, under LaTeX2e
\addtolength{\oddsidemargin}{-.6in} 
\addtolength{\evensidemargin}{-.6in}
\addtolength{\topmargin}{-.4in}
\addtolength{\textwidth}{1.2in}
\addtolength{\textheight}{.6in}

\renewcommand{\baselinestretch}{1.05}

\usepackage{verbatim,fancyvrb}

\usepackage{palatino}

\newtheorem*{thm}{Theorem}
\newtheorem*{defn}{Definition}
\newtheorem*{example}{Example}
\newtheorem*{problem}{Problem}
\newtheorem*{remark}{Remark}

\newcommand{\mtt}{\texttt}
\usepackage{alltt,xspace}
\newcommand{\mfile}[1]
{\medskip\begin{quote}\scriptsize \begin{alltt}\input{#1.m}\end{alltt} \normalsize\end{quote}\medskip}

\usepackage[final]{graphicx}
\newcommand{\mfigure}[1]{\includegraphics[height=2.5in,
width=3.5in]{#1.eps}}
\newcommand{\regfigure}[2]{\includegraphics[height=#2in,
keepaspectratio=true]{#1.eps}}
\newcommand{\widefigure}[3]{\includegraphics[height=#2in,
width=#3in]{#1.eps}}

\usepackage{amssymb}

\usepackage[pdftex, colorlinks=true, plainpages=false, linkcolor=black, citecolor=red, urlcolor=red]{hyperref}

% macros
\newcommand{\br}{\mathbf{r}}
\newcommand{\bv}{\mathbf{v}}
\newcommand{\bx}{\mathbf{x}}
\newcommand{\by}{\mathbf{y}}

\newcommand{\CC}{\mathbb{C}}
\newcommand{\RR}{\mathbb{R}}
\newcommand{\ZZ}{\mathbb{Z}}

\newcommand{\eps}{\epsilon}
\newcommand{\grad}{\nabla}
\newcommand{\lam}{\lambda}
\newcommand{\lap}{\triangle}

\newcommand{\ip}[2]{\ensuremath{\left<#1,#2\right>}}

%\renewcommand{\det}{\operatorname{det}}
\newcommand{\onull}{\operatorname{null}}
\newcommand{\rank}{\operatorname{rank}}
\newcommand{\range}{\operatorname{range}}

\newcommand{\Julia}{\textsc{Julia}\xspace}
\newcommand{\Matlab}{\textsc{Matlab}\xspace}
\newcommand{\Octave}{\textsc{Octave}\xspace}
\newcommand{\Python}{\textsc{Python}\xspace}

\newcommand{\prob}[1]{\bigskip\noindent\textbf{#1}\quad }

\newcommand{\chapexers}[2]{\prob{Chapter #1, pages #2, Exercises:}}
\newcommand{\exer}[2]{\prob{Exercise #1}}

\newcommand{\pts}[1]{(\emph{#1 pts}) }
\newcommand{\epart}[1]{\medskip\noindent\textbf{(#1)}\quad }
\newcommand{\ppart}[1]{\,\textbf{(#1)}\quad }

\newcommand*\circled[1]{\tikz[baseline=(char.base)]{
            \node[shape=ellipse,draw,inner sep=2pt] (char) {#1};}}


\begin{document}
\scriptsize \noindent Math 661 Optimization (Bueler) \hfill 19 September, 2022
\normalsize

\medskip\bigskip

\Large\centerline{\textbf{Assignment \#3}}
\large
\bigskip

\centerline{\textbf{Due Wednesday, 28 September 2022, at the start of class}}
\bigskip
\normalsize

\thispagestyle{empty}

\bigskip
From the textbook\footnote{Griva, Nash, and Sofer, \emph{Linear and Nonlinear Optimization}, 2nd ed., SIAM Press 2009.} please read sections 2.4, 2.6, and 3.1, plus Appendices B.4 through B.8.  We will get back to sections 2.5 (``rates of convergence'') and 2.7 (``Newton's method for nonlinear equations'') soon enough, but we will be pushing toward the simplex method for linear programming in the shorter term (Chapters 3,4,5).

\bigskip
\noindent \textsc{Do the following exercise} from section 2.4, page 58:

\begin{itemize}
\item Exercise 4.3
\end{itemize}

\bigskip
\noindent \textsc{Do the following exercises} from section 2.6, page 66:

\begin{itemize}
\item Exercise 6.5
\item Exercise 6.6
\end{itemize}

\bigskip
\noindent \textsc{Do the following exercises} from section 3.1, page 82:

\begin{itemize}
\item Exercise 1.1
\item Exercise 1.3
\end{itemize}

\prob{Problem P8.}  \emph{This problem considers a simple, \textbf{\emph{and totally inadequate}}, algorithm to illustrate ``General Optimization Algorithm II'' on page 55.  We will do much better soon, but the reason I am asking this weird question is that some of the difficulties seen here will remain for all time.\footnote{Optimization is not easy, and algorithms always require the user to know \emph{some} properties of their problem.}  This exercise asks you to analyze the deficiencies of this particular algorithm, \emph{not} propose a better one!\footnote{I would, in fact, expect that you can propose a better one.}}

Consider one-dimensional unconstrained optimization problems
    $$\min_{x\in \RR} f(x)$$
for objective functions $f$ which have one continuous derivative $f'$.  I propose the following algorithm:

\medskip
\fbox{\parbox{\textwidth}{\begin{quote}
\textbf{Algorithm P8A.}  Assume functions $f(x)$ and $f'(x)$ are supplied.  
\renewcommand{\labelenumi}{\arabic{enumi}.}
\begin{enumerate}
\item Set $x_0=1$.
\item For $k=0,1,2,\dots$
    \renewcommand{\labelenumii}{(\roman{enumii})}
    \begin{enumerate}
    \item If $|f'(x_k)| < 10^{-3}$ then stop.
    \item If $f'(x_k) > 0$ then let $p_k=-1$.  Otherwise let $p_k=+1$.
    \item Let $\alpha_k=0.01$.  Let $x_{k+1}=x_k+\alpha_k p_k$.
    \end{enumerate}
\end{enumerate}
\end{quote}}}

\bigskip
\renewcommand{\labelenumi}{\textbf{\alph{enumi})}}
\begin{enumerate}
\item Implement Algorithm P8A.  In \Matlab it will be a function

    \centerline{\texttt{function z = p8a(f,df)}}

\noindent where the inputs are functions $f=$ \texttt{f} and $f'=$ \texttt{df}.  The output \texttt{z} is the proposed $x$-coordinate of the solution, i.e.~it is the minimizer.

\bigskip

\item Run the algorithm, and state briefly what happens, in the following cases:
    \renewcommand{\labelenumii}{(\roman{enumii})}
    \begin{enumerate}
    \item $f(x) = x^2 - 3 x + 2$
    \item $f(x) = \cos(x/50)$
    \item $f(x) = e^{\sin(10 x)}$
    \item $f(x) = \operatorname{sech}(x)$
    \end{enumerate}

\medskip

\item In several complete sentences, describe the main deficiencies of Algorithm P8A.  It is a good idea to use the results of part \textbf{b)} to illustrate some of your points.

\medskip

\item Next, observe that there are magic, fixed numbers inside Algorithm P8A which we can instead put under the control of the user.  This gives a new algorithm:

\medskip
\fbox{\parbox{\textwidth}{\begin{quote}
\textbf{Algorithm P8B.}  Assume functions $f(x)$ and $f'(x)$ and numbers $x_0\in \RR$, $\eps>0$, and $\delta >0$ are supplied.
\renewcommand{\labelenumii}{\arabic{enumii}.}
\begin{enumerate}
\item (\emph{The user has supplied $x_0$.})
\item For $k=0,1,2,\dots$
    \renewcommand{\labelenumiii}{(\roman{enumiii})}
    \begin{enumerate}
    \item If $|f'(x_k)| < \eps$ then stop.
    \item If $f'(x_k) > 0$ then let $p_k=-1$.  Otherwise let $p_k=+1$.
    \item Let $\alpha_k=\delta$.  Let $x_{k+1}=x_k+\alpha_k p_k$.
    \end{enumerate}
\end{enumerate}
\end{quote}}}
\medskip

\noindent There is no request to implement P8B.  However, in \Matlab it would be a function like

    \centerline{\texttt{function z = p8b(f,df,x0,eps,delta)}}

\noindent In complete sentences, address whether Algorithm P8B is significantly better than P8A.  Specifically, consider what the user needs to know about $f(x)$ in order to use P8B effectively.

\end{enumerate}

\end{document}
