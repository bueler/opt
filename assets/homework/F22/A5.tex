\documentclass[12pt]{amsart}
%prepared in AMSLaTeX, under LaTeX2e
\addtolength{\oddsidemargin}{-.6in} 
\addtolength{\evensidemargin}{-.6in}
\addtolength{\topmargin}{-.4in}
\addtolength{\textwidth}{1.2in}
\addtolength{\textheight}{.6in}

\renewcommand{\baselinestretch}{1.05}

\usepackage{verbatim,fancyvrb}

\usepackage{palatino}

\newtheorem*{thm}{Theorem}
\newtheorem*{defn}{Definition}
\newtheorem*{example}{Example}
\newtheorem*{problem}{Problem}
\newtheorem*{remark}{Remark}

\newcommand{\mtt}{\texttt}
\usepackage{alltt,xspace}
\newcommand{\mfile}[1]
{\medskip\begin{quote}\scriptsize \begin{alltt}\input{#1.m}\end{alltt} \normalsize\end{quote}\medskip}

\usepackage[final]{graphicx}
\newcommand{\mfigure}[1]{\includegraphics[height=2.5in,
width=3.5in]{#1.eps}}
\newcommand{\regfigure}[2]{\includegraphics[height=#2in,
keepaspectratio=true]{#1.eps}}
\newcommand{\widefigure}[3]{\includegraphics[height=#2in,
width=#3in]{#1.eps}}

\usepackage{amssymb}

\usepackage[pdftex, colorlinks=true, plainpages=false, linkcolor=black, citecolor=red, urlcolor=red]{hyperref}

% macros
\newcommand{\br}{\mathbf{r}}
\newcommand{\bv}{\mathbf{v}}
\newcommand{\bx}{\mathbf{x}}
\newcommand{\by}{\mathbf{y}}

\newcommand{\CC}{\mathbb{C}}
\newcommand{\RR}{\mathbb{R}}
\newcommand{\ZZ}{\mathbb{Z}}

\newcommand{\eps}{\epsilon}
\newcommand{\grad}{\nabla}
\newcommand{\lam}{\lambda}
\newcommand{\lap}{\triangle}

\newcommand{\ip}[2]{\ensuremath{\left<#1,#2\right>}}

%\renewcommand{\det}{\operatorname{det}}
\newcommand{\onull}{\operatorname{null}}
\newcommand{\rank}{\operatorname{rank}}
\newcommand{\range}{\operatorname{range}}

\newcommand{\Julia}{\textsc{Julia}\xspace}
\newcommand{\Matlab}{\textsc{Matlab}\xspace}
\newcommand{\Octave}{\textsc{Octave}\xspace}
\newcommand{\Python}{\textsc{Python}\xspace}

\newcommand{\prob}[1]{\bigskip\noindent\textbf{#1}\quad }

\newcommand{\chapexers}[2]{\prob{Chapter #1, pages #2, Exercises:}}
\newcommand{\exer}[2]{\prob{Exercise #1}}

\newcommand{\pts}[1]{(\emph{#1 pts}) }
\newcommand{\epart}[1]{\medskip\noindent\textbf{(#1)}\quad }
\newcommand{\ppart}[1]{\,\textbf{(#1)}\quad }

\newcommand*\circled[1]{\tikz[baseline=(char.base)]{
            \node[shape=ellipse,draw,inner sep=2pt] (char) {#1};}}


\begin{document}
\scriptsize \noindent Math 661 Optimization (Bueler) \hfill 5 October, 2022
\normalsize

\medskip\bigskip

\Large\centerline{\textbf{Assignment \#5}}
\large
\bigskip

\centerline{\textbf{Due Monday, 17 October 2022, at the start of class}}
\bigskip
\normalsize

\thispagestyle{empty}

\bigskip
From the textbook\footnote{Griva, Nash, and Sofer, \emph{Linear and Nonlinear Optimization}, 2nd ed., SIAM Press 2009.} please read Chapter 4 and sections 5.1--5.2, but note that you can skip subsections 5.2.3 and 5.2.4 on tableaus.

\bigskip
\noindent \textsc{Do the following exercises} from section 4.2, pages 105--106:

\begin{itemize}
\item Exercise 2.1
\item Exercise 2.2
\item Exercise 2.4
\end{itemize}

\bigskip
\noindent \textsc{Do the following exercises} from section 4.3, pages 114--116:

\begin{itemize}
\item Exercise 3.1
\item Exercise 3.12
\end{itemize}

\bigskip
\noindent \textsc{Do the following exercises} from section 5.2, pages 141--142.  For \emph{each} of these problems print out a FIXME \href{FIXME}{template} and fill it in by hand:

\begin{itemize}
\item Exercise 2.2 (i)
\item Exercise 2.2 (iii)
\item Exercise 2.2 (vi)
\end{itemize}


\prob{Problem P9.}  On pages 90--91 the book describes how to use the QR decomposition to build a null-space matrix for $A$ in a numerically-stable way:

\begin{quote}
\dots let $A$ be an $m\times n$ matrix with full row rank.  We perform an orthogonal factorization of $A^\top$:
    $$A^\top = QR.$$
[Then let] $Q=(Q_1,Q_2)$, where $Q_1$ consists of the first $m$ columns of $Q$ and $Q_2$ consists of the last $n-m$ columns.  [Then]
    $$Z = Q_2$$
\end{quote}

\noindent Note that an $m\times n$ matrix with full \emph{row} rank has $m\le n$, so in the description above $n-m$ is either zero or positive.  As the book says, the columns of $Z$ are not just a basis for the null space $\mathcal{N}(A)$, but a well-behaved \emph{orthogonal} basis for $\mathcal{N}(A)$.

Write a \Matlab function\footnote{In Python, functions \texttt{qr()} and \texttt{null\_space()} from \texttt{scipy.linalg} replace \Matlab commands \texttt{qr()} and \texttt{null()} in this problem.  In Julia use \texttt{qr()} and \texttt{nullspace()}.}

\centerline{\texttt{function Z = mynull(A)}}

\noindent which implements the above strategy.  In \Matlab the ``orthogonal factorization'' step can use the function \verb|qr()|; you do not have to worry how \verb|qr()| works.  Your code should be quite short.  Note that \verb|size(A)| will tell you the values of $m$ and $n$.  Your code should stop with an error if $m>n$.

Test your \verb|mynull()| on the matrices
    $$A_1 = \begin{pmatrix}
    1 & 2 & 3 & 4 & 5 \\
    6 & 7 & 8 & 9 & 10 \\
    4 & 1 & 0 & 1 & 4
    \end{pmatrix}$$
    $$A_2 = \begin{pmatrix} 0 & 1 & 2 & 0 \end{pmatrix}$$
    $$A_3 = \begin{pmatrix} 1 & 2 \\ 3 & 4 \end{pmatrix}.$$
Are the columns of $Z$ actually in the null space of the matrix $A_j$ in each case?  (\emph{Show command-line \Matlab verifications.})

How does the result of \verb|mynull()| differ from the result of the built-in command \verb|null()| on the above matrices?  (\emph{Use} \verb|norm| \emph{to answer this}.)  Is \verb|null()| implemented the same way as \verb|mynull()|?

\end{document}
