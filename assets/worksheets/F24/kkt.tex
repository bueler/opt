\documentclass[11pt]{amsart}
%prepared in AMSLaTeX, under LaTeX2e
\addtolength{\oddsidemargin}{-.75in} 
\addtolength{\evensidemargin}{-.75in}
\addtolength{\topmargin}{-.4in}
\addtolength{\textwidth}{1.4in}
\addtolength{\textheight}{1.0in}

\renewcommand{\baselinestretch}{1.075}

\usepackage{verbatim,fancyvrb}

\usepackage{palatino,amssymb}

\usepackage{tikz}
\usetikzlibrary{arrows.meta}

\newtheorem*{thm}{Theorem}
\newtheorem*{defn}{Definition}

\theoremstyle{definition}
\newtheorem*{example}{Example}
\newtheorem*{problem}{Problem}
\newtheorem*{remark}{Remark}

\newcommand{\mtt}{\texttt}
\usepackage{alltt,xspace}
\newcommand{\mfile}[1]
{\medskip\begin{quote}\scriptsize \begin{alltt}\input{#1.m}\end{alltt} \normalsize\end{quote}\medskip}

%\usepackage[final]{graphicx}

\usepackage[pdftex, colorlinks=true, plainpages=false, linkcolor=blue, citecolor=red, urlcolor=blue]{hyperref}

% macros
\newcommand{\bc}{\mathbf{c}}
\newcommand{\br}{\mathbf{r}}
\newcommand{\bv}{\mathbf{v}}
\newcommand{\bx}{\mathbf{x}}
\newcommand{\by}{\mathbf{y}}

\newcommand{\CC}{\mathbb{C}}
\newcommand{\RR}{\mathbb{R}}
\newcommand{\ZZ}{\mathbb{Z}}

\newcommand{\eps}{\epsilon}
\newcommand{\grad}{\nabla}
\newcommand{\lam}{\lambda}
\newcommand{\lap}{\triangle}

\newcommand{\ip}[2]{\ensuremath{\left<#1,#2\right>}}

%\renewcommand{\det}{\operatorname{det}}
\newcommand{\onull}{\operatorname{null}}
\newcommand{\rank}{\operatorname{rank}}
\newcommand{\range}{\operatorname{range}}

\newcommand{\prob}[1]{\bigskip\noindent\textbf{#1.}\quad }
\newcommand{\exer}[2]{\prob{Exercise #2 in Lecture #1}}

\newcommand{\Julia}{\textsc{Julia}\xspace}
\newcommand{\Matlab}{\textsc{Matlab}\xspace}
\newcommand{\Octave}{\textsc{Octave}\xspace}
\newcommand{\Python}{\textsc{Python}\xspace}

\DefineVerbatimEnvironment{mVerb}{Verbatim}{numbersep=2mm,
frame=lines,framerule=0.1mm,framesep=2mm,xleftmargin=4mm,fontsize=\footnotesize}

\newcommand{\ema}{\emach}
\newcommand{\emach}{\eps_{\!_{\text{m}}}}

\newcommand{\ppart}[1]{\quad \textbf{(#1)} }
\newcommand{\epart}[1]{\medskip\noindent\textbf{(#1)} \quad}


\begin{document}
\scriptsize \noindent Math 661 Optimization (Bueler) \hfill 25 November 2024
\normalsize

\medskip\bigskip
\Large
\centerline{KKT conditions for general nonlinear optimization}

\bigskip\medskip
\normalsize

\thispagestyle{empty}

These short notes are about the most general problem considered in this course, a nonlinear constrained optimization problem over $x\in\RR^n$:
\begin{alignat*}{2}
    \text{minimize}   &&  &f(x) \\
    \text{subject to} && \qquad g_i(x) &= 0, \quad i=1,\dots,\ell \\
                      &&       h_i(x) &\ge 0, \quad i=1,\dots,m
\end{alignat*}
We will assume that all functions $f,g_i,h_i$ are as differentiable as needed.

Define $g(x)$ to be the vector formed from $g_1(x),\dots,g_\ell(x)$, and $h(x)$ from $h_1(x),\dots,h_m(x)$.  In these terms we can write
\begin{equation}
\begin{matrix}
    \text{minimize}   & f(x) \\
    \text{subject to} & g(x) = 0 \\
                      & h(x) \ge 0
\end{matrix} \label{prob}
\end{equation}

We will need a definition which is given in section 14.5 of the textbook.\footnote{Griva, Nash, and Sofer, \emph{Linear and Nonlinear Optimization}, 2nd ed., SIAM Press 2009.}  For a feasible point $x$, let $\tilde h(x)$ be the vector of length $\tilde m$ formed from the \emph{active} constraints $h_i(x)$ at $x$, i.e.~for which $h_i(x)=0$.

\begin{defn}  A feasible point $x_*$ is a \emph{regular point} of the constraints if the matrix
    $$\begin{bmatrix}
    \grad g_1(x_*) & \dots & \grad g_\ell(x_*) & \grad {\tilde h}_1(x_*) & \dots & \grad {\tilde h}_{\tilde m}(x_*)
    \end{bmatrix}$$
has linearly independent columns.  This matrix has $n$ rows and $\ell + \tilde m$ columns.  (Here the gradient of a scalar-valued function is a column vector.)
\end{defn}

\noindent Conceptually, at a regular point each active constraint provides new information; there are no redundancies.  Note that the inactive inequality constraints are not relevant in this definition.  This definition is called the \emph{linearly independent constraint qualification (LIQC)} in some books.

The Lagrangian for the problem is
\begin{align*}
\mathcal{L}(x,\lambda,\mu) &= f(x) - \sum_{i=1}^\ell \lambda_i g_i(x) - \sum_{j=1}^m \mu_j h_j(x) \\
  &= f(x) - \lambda^\top g(x) - \mu^\top h(x)
\end{align*}
where $\lambda\in\RR^\ell$ and $\mu\in\RR^m$ are column vectors.  In the second form we must conceive of $g(x)$ and $h(x)$ as column vectors.  Note that $h(x)$, not $\tilde h(x)$, is used here.

The following KKT theorem\footnote{\href{https://en.wikipedia.org/wiki/Karush-Kuhn-Tucker_conditions}{\texttt{en.wikipedia.org/wiki/Karush-Kuhn-Tucker\_conditions}}} states the first-order necessary conditions.

\begin{thm}[Karush-Kuhn-Tucker, 1939 \& 1951]  Suppose $x_*$ is a local minimizer of problem \eqref{prob}, and assume it is a regular point of the constraints.  Then there exist vectors $\lambda_*\in\RR^\ell$ and $\mu_*\in\RR^m$ so that
\begin{align*}
\grad_x \mathcal{L}(x_*,\lambda_*,\mu_*) &= 0 &&\text{stationarity} \\
g(x_*) &= 0  &&\text{primal feasibility: equality constraints} \\
h(x_*) &\ge 0  &&\text{primal feasibility: inequality constraints} \\
\mu_* &\ge 0 &&\text{dual feasibility} \\
\mu_*^\top h(x_*) &= 0 &&\text{complementary slackness}
\end{align*}
\end{thm}

\medskip
\noindent The stationarity condition can be written
\begin{align*}
\grad f(x_*) &= \grad g(x_*) \lambda_* + \grad h(x_*) \mu_* \\
  &= (\lambda_*)_1 \grad g_1(x_*) + \dots + (\lambda_*)_\ell \grad g_\ell(x_*) + (\mu_*)_1 \grad h_1(x_*) + \dots + (\mu_*)_m \grad h_m(x_*)
\end{align*}
That is, the gradient of $f$ at the solution can be written as a linear combination of the gradients of the constraints.  For the inequality constraints, however, complementary slackness will cause some of the multipliers $\mu_i$ to be zero.  The fact that $x_*$ is a regular point implies that the linear combination is unique, thus that the Lagrange multipliers $\lambda_*,\mu_*$ are unique.

It will be helpful to do an example with both equality and inequality constraints, and with more than one inequality constraint so we can see complementary slackness in action.

\begin{example}
    $$\begin{matrix}
    \text{minimize}   & f(x) = \frac{1}{2} \left(x_1^2 + x_2^2\right) \\
    \text{subject to} & x_1 + x_2 = 1 \\
                      & x_1 \ge 1 \\
                      & x_2 \ge -1
\end{matrix}$$
Here $g_1(x) = x_1 + x_2 -1$, $h_1(x) = x_1 - 1$, and $h_2(x) = x_2 + 1$.  The Lagrangian is
	$$\mathcal{L}(x,\lambda,\mu) = \frac{1}{2} \left(x_1^2 + x_2^2\right) - \lambda_1(x_1 + x_2 - 1) - \mu_1 (x_1 - 1) - \mu_2(x_2 + 1)$$
and 
The 5 conditions of the KKT theorem are now 8 facts that must be true at the solution:
\begin{align*}
x_1 - \lambda_1 - \mu_1 &= 0 &&\text{\emph{stationarity}} \\
x_2 - \lambda_1 - \mu_2 &= 0 \\
x_1 + x_2 -1 &= 0  &&\text{\emph{primal feasibility: equality constraint}} \\
x_1 - 1 &\ge 0  &&\text{\emph{primal feasibility: inequality constraints}} \\
x_2 + 1 &\ge 0 \\
\mu_1 &\ge 0 &&\text{\emph{dual feasibility}} \\
\mu_2 &\ge 0 \\
\mu_1 (x_1 - 1) + \mu_2 (x_2 + 1) &= 0 &&\text{\emph{complementary slackness}}
\end{align*}
On the other hand, if you draw the $x_1,x_2$ plane then the feasible set is the closed line segment between the points $(1,0)$ and $(2,-1)$, and the objective is essentially the square of the distance to the origin.  So we believe that $x_*=(1,0)$ is the solution, and that the first inequality constraint is active while the second is not.   This means that $\mu_2=0$.  The complementary slackness condition is $\mu_1 (x_1-1)=0$, so $x_1=1$ assuming $\mu_1>0$, which we will check.  The system of the remaining \emph{equations} is now reduced to
\begin{align*}
1 - \lambda_1 - \mu_1 &= 0 \\
x_2 - \lambda_1 &= 0 \\
x_2 &= 0 \\
\end{align*}
which is a set of 3 equations for $x_2,\lambda_1,\mu_1$.  It is easy to see $x_2=0$, $\lambda_1=0$, and $\mu_1=1$.  That is,
   $$(x_1,x_2,\lambda_1,\mu_1,\mu_2) = (1,0,0,1,0).$$
It is easy to check that with these values all 8 KKT conditions hold.  In particular, complementary slackness says $1 \cdot 0 + 0 \cdot 1 = 0$, and strict complementarity holds.
\end{example}

\end{document}

