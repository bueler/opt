\documentclass[11pt]{amsart}
%prepared in AMSLaTeX, under LaTeX2e
\addtolength{\oddsidemargin}{-.75in} 
\addtolength{\evensidemargin}{-.75in}
\addtolength{\topmargin}{-.4in}
\addtolength{\textwidth}{1.4in}
\addtolength{\textheight}{1.0in}

\renewcommand{\baselinestretch}{1.075}

\usepackage{verbatim,fancyvrb}

\usepackage{palatino,amssymb}

\usepackage{tikz}
\usetikzlibrary{arrows.meta}

\newtheorem*{thm}{Theorem 16.0}
\newtheorem*{defn}{Definition}
\newtheorem*{example}{Example}
\newtheorem*{problem}{Problem}
\newtheorem*{remark}{Remark}

\newcommand{\mtt}{\texttt}
\usepackage{alltt,xspace}
\newcommand{\mfile}[1]
{\medskip\begin{quote}\scriptsize \begin{alltt}\input{#1.m}\end{alltt} \normalsize\end{quote}\medskip}

%\usepackage[final]{graphicx}

\usepackage[pdftex, colorlinks=true, plainpages=false, linkcolor=blue, citecolor=red, urlcolor=blue]{hyperref}

% macros
\newcommand{\bc}{\mathbf{c}}
\newcommand{\br}{\mathbf{r}}
\newcommand{\bv}{\mathbf{v}}
\newcommand{\bx}{\mathbf{x}}
\newcommand{\by}{\mathbf{y}}

\newcommand{\CC}{\mathbb{C}}
\newcommand{\RR}{\mathbb{R}}
\newcommand{\ZZ}{\mathbb{Z}}

\newcommand{\eps}{\epsilon}
\newcommand{\grad}{\nabla}
\newcommand{\lam}{\lambda}
\newcommand{\lap}{\triangle}

\newcommand{\ip}[2]{\ensuremath{\left<#1,#2\right>}}

%\renewcommand{\det}{\operatorname{det}}
\newcommand{\onull}{\operatorname{null}}
\newcommand{\rank}{\operatorname{rank}}
\newcommand{\range}{\operatorname{range}}

\newcommand{\prob}[1]{\bigskip\noindent\textbf{#1.}\quad }
\newcommand{\exer}[2]{\prob{Exercise #2 in Lecture #1}}

\newcommand{\pts}[1]{(\emph{#1 pts}) }
\newcommand{\epart}[1]{\medskip\noindent\textbf{(#1)}\quad }
\newcommand{\ppart}[1]{\,\textbf{(#1)}\quad }

\newcommand{\Julia}{\textsc{Julia}\xspace}
\newcommand{\Matlab}{\textsc{Matlab}\xspace}
\newcommand{\Octave}{\textsc{Octave}\xspace}
\newcommand{\Python}{\textsc{Python}\xspace}

\DefineVerbatimEnvironment{mVerb}{Verbatim}{numbersep=2mm,
frame=lines,framerule=0.1mm,framesep=2mm,xleftmargin=4mm,fontsize=\footnotesize}

\newcommand{\ema}{\emach}
\newcommand{\emach}{\eps_{\!_{\text{m}}}}


\begin{document}
\scriptsize \noindent Math 661 Optimization (Bueler) \hfill 29 August, 2022
\normalsize

\medskip\bigskip
\Large
\centerline{5 example optimization problems}

\bigskip\medskip
\normalsize

\thispagestyle{empty}

I have several goals in starting this course with examples:
\begin{itemize}
\item It shows how optimization comes from real-world applications.
\item It allows you to discover for yourself some basic theoretical/numerical ideas.
\item These examples help you practice and learn \Matlab or other\footnote{In this course you may use other languages such as \Python or \Julia, but I will provide demos, examples, and solutions in \Matlab.  Note that a \Matlab code should work in \Octave and vice versa.} programming.
\end{itemize}
The textbook, namely Griva, Nash, and Sofer, \emph{Linear and Nonlinear Optimization}, 2nd ed., SIAM Press 2009, also provides many examples; see Chapter 1.  You will not be able to understand our theory and algorithms without some understanding of applications like these.  Indeed, all optimization experts have learned from such examples.

Following the ideas in Chapter 2.1 in the textbook, each example identifies a \emph{feasible set} $S$ and an \emph{objective function} $f(x)$, so each example is written in this standard form:
    $$\min_{x\in S} f(x).$$

This document does \emph{not} address how to solve these problems!  That will be done by you; see Assignment \#1 for specific expectations.  When you solve one of these problems your method may, and often will, be ``brute force'' and inefficient.  That is just fine for now!  The rest of the course will make more sense if you see brute force approaches before more elegant algorithms.

Each example has a name written in parentheses.  I will also use this name for my \Matlab code (e.g.~\texttt{calcone.m}) when I hand out solutions.

\bigskip
\renewcommand{\labelenumi}{\arabic{enumi}. \quad}
\begin{enumerate}
\item (\texttt{calcone})  \quad Let
    $$f(x) = \left(x^2 + \sin x\right)^2 - 10 \left(\cos(5 x) + \frac{3}{2} x\right).$$
Compute the minimum of $f$ on the interval $S=[0,2]$:
    $$\min_{x\in [0,2]} f(x)$$

You saw such problems in Calculus I, but this one is hard to do by hand.  It benefits from computer visualization, and, because $S$ is one-dimensional, you may easily plot $f(x)$ on $S$.  From such a plot you can get close to the solution just by looking.


\bigskip
\item (\texttt{fit})  \quad Consider the following 11 data points which are plotted below:

\bigskip
\begin{tabular}{c|ccccccccccc}
x & 0.000 & 0.100 & 0.200 & 0.300 & 0.400 & 0.500 & 0.600 & 0.700 &  0.800 &  0.900 &  1.000 \\
\hline
y & 4.914 & 3.666 & 2.289 & 1.655 & 1.029 & 0.739 & 0.393 & 0.090 & -0.197 & -0.721 & -0.971
\end{tabular}

\bigskip
\begin{center}
\includegraphics[width=0.45\textwidth]{fitdata}
\end{center}

\medskip
Suppose we believe that this data can be fit by a function of the form
    $$g(x) = c_1 + c_2 x + c_3 e^{-5 x}.$$
Let's decide that the meaning of ``fit'' is that the sum of the squares of the misfits should be as small as possible.  Then the problem is
	$$\min_{c \in \RR^3} f(c)$$
where we define the objective function 
	$$f(c) = \frac{1}{2} \sum_{j=1}^{11} \left(g(x_j) - y_j\right)^2 = \frac{1}{2} \sum_{j=1}^{11} \left(c_1 + c_2 x_j + c_3 e^{-5 x_j} - y_j\right)^2.$$
(The overall factor of $1/2$ is merely a convenience when differentiating.)  Note $S=\RR^3$ because there are no constraints on the coefficients $c_i$.

We are \emph{not} finding $x_j$ or $y_j$ values in the minimization process!  We are finding $c_1,c_2,c_3$.  Note the data values $(x_j,y_j)$ determine the objective function.

\bigskip
\item (\texttt{salmon})  \quad Ed and Vera caught 22 salmon.  Of these, $x_1$ will be eaten fresh, which requires 2 time units per fish.  Then $x_2$ will be vacuum-packed and frozen (3 time units per fish) and another $x_3$ will be smoked and vacuum-packed (4 time units per fish).  Thus the total amount of processing time is $f(x) = 2 x_1 + 3 x_2 + 4 x_3$.  However, at most 2 fish can be eaten fresh before they go bad, at most 10 fish can be smoked in the smoker time available, and at least 4 fish must be smoked because they'll be mailed unfrozen to relatives.  Find $x_1,x_2,x_3$ to minimize the total processing time.

This is a constrained minimization problem wherein $x_i$ are numbers of fish, \emph{which must be positive numbers}, and the objective function is the total processing time:
	$$\min f(x) \qquad \text{subject to }\quad \begin{matrix} x_1 + x_2 + x_3 = 22 \\ 0 \le x_1 \le 2 \\ 0 \le x_2 \\ 4 \le x_3 \le 10 \end{matrix}$$
The feasible set $S\subset \RR^3$ includes all the constraints:
    $$S = \left\{x \in \RR^3\,\big|\,x_1 + x_2 + x_3 = 22, 0 \le x_1 \le 2, 0 \le x_2, \text{ and } 4 \le x_3 \le 10\right\}.$$
Note that the objective function and the constraint functions (e.g.~$g_1(x)=x_1+x_2+x_3$, $g_2(x)=x_1$, \dots) are linear functions, so this is a \emph{linear programming} problem.

\clearpage \newpage
\item (\texttt{tsp})  \quad Jill sells amazing widgets that help you learn math.  To sell these devices she plans to visit six cities A, F, J, N, W by starting and ending at city S.  Some cities have connecting flights and some do not; the one-way costs of the various flights are shown below in a \emph{graph} with costs (weights) on each connection (edge).  Except for starting and ending at S, it is clear that she should visit each city exactly once.

\begin{center}
\vspace{-5mm}
\begin{tikzpicture}[scale=0.9]
\begin{scope}[every node/.style={circle,thick,draw}]
    \node (A) at (-0.5,0) {A};
    \node (F) at (0,3)    {F};
    \node (J) at (2,-1)   {J};
    \node (N) at (-3,3)   {N};
    \node (S) at (3,-3)   {S};
    \node (W) at (3,1)    {W} ;
\end{scope}

\begin{scope}[every node/.style={fill=white},
              every edge/.style={draw=black,very thick}]
    \path (A) edge node {$100$} (F);
    \path (A) edge node {$100$} (J);
    \path (A) edge node {$150$} (N);
    \path (A) edge[bend right=50] node {$250$} (S);
    \path (A) edge node {$150$} (W);
    %\path (F) edge node {$150$} (J);
    \draw[very thick] (F.east) .. controls (3,4) and (6,0) .. (J.east) node[midway] {$150$};
    \path (F) edge node {$200$} (N);
    %\path (F) edge node {$300$} (S);
    \draw[very thick] ([yshift=3mm,xshift=-1mm] F.east) .. controls (4,5) and (7,-1) .. (S.east) node[midway] {$300$};
    \path (F) edge node {$250$} (W);
    \path (J) edge node {$200$} (S);
    \path (J) edge node {$200$} (W);
\end{scope}
\end{tikzpicture}
\end{center}

\medskip
This is an example of the famous \emph{traveling salesperson problem}.  Each possible itinerary is expressible as a seven-letter string like ``SANFWJS.''  If $x$ denotes such a feasible string then we may define the objective function $f(x)$ to be the cost of that itinerary; thus $f(x)$ is defined using the edge weights.  Even finding a feasible itinerary, a \emph{Hamiltonian cycle}, for a big-enough graph, is generally nontrivial.  One may, however, add-in all missing edges with large weights so that any itinerary $x$ is feasible and has a well-defined cost $f(x)$.

The problem \emph{could} be written in standard form
	$$\min_S f(x).$$
where $S = \{x\,|\,x \text{ is a feasible itinerary}\}$.  However, there is no easy way to describe $S$ by inequalities and equalities as as subset of some Euclidean space $\RR^n$ as above.

In any case, this is a \emph{discrete optimization} problem.  That is, $S$ is a finite set of feasible itineraries.  Mostly we will consider continuous optimization problems in this course, not discrete ones.

\bigskip
\item (\texttt{glacier})  \quad  The shape of a glacier on flat bedrock is approximately given by the solution to a constrained optimization problem.  The mathematical form of this problem must express three ideas.  First that glaciers are created where snowfall exceeds melt, second that their shapes are determined by the fact that glaciers also flow downhill, and finally that the thickness of a glacier is a nonnegative function.

Even in the simple case here, the solution to the minimization problem is itself a \emph{function}.  In fact the objective function $f[u]$ takes a function $u(x)$ as input and produces a single real number.  It is common to call such functions, which take other functions as inputs, \emph{functionals}.

The feasible set in this specific problem is a set of functions defined on an interval:
    $$S = \left\{u(x) \,\big|\, u(x) \ge 0 \text{ is a differentiable function on } [-100,100]\right\}.$$
Note $x$ is in kilometers.  If $u(x)$ is in $S$ then it is nonnegative; this nontrivial constraint is needed because $u(x)$ is a power of the ice thickness (see below).  In contrast to all the above examples, an element of the feasible set $S$ is not a finite list (vector) of numbers, it is a function itself.  Therefore this \emph{calculus of variations} problem is infinite-dimensional.

In this specific problem I will make up a function which defines the rate of total snowfall or melt in a year, the \emph{mass balance} in glaciologist language:
    $$m(x) = \begin{cases} -3 + \frac{6}{90} (x+70), & -70 \le x \le 20, \\
                            3 - \frac{6}{50} (x-20), & 20 \le x \le 70, \\
                           -3, & \text{otherwise}.
             \end{cases}$$
The units of $m(x)$ are meters per year.  This function is graphed below.  Note that it is only snowing where $m(x)$ is positive; everywhere else it is melting.

\bigskip
\begin{center}
\includegraphics[width=0.5\textwidth]{massbalance}  % see massbalance.m
\end{center}

% regarding how mu is computed:
%>> rho = 910; g = 9.81; p = 4;
%>> A = 1e-16/31556926
%A =    3.1689e-24
%>> Lam = 2 * (rho*g)^3 *A / 5
%Lam =    9.0177e-13
%>> mu = 2 * (rho * g * (p-1) / (2*p))^(p-1) * A / (p+1)
%mu =    4.7554e-14

The objective functional is an integral defined using the data $m(x)$:
    $$f[u] = \int_{-100}^{100} \frac{\mu}{4} \left(u'(x)\right)^4 - m(x) u(x)\,dx$$
Based on other physical constants related to the flow of ice (not shown) we set $\mu = 5 \times 10^{-14}$.  The glacier shape is derived from the solution of the problem
    $$\min_S f[u].$$
Once $u(x)$ is computed we raise it to a power to get the actual height $h(x)$, measured in meters, of the glacier:
    $$h(x) = u(x)^{3/8}.$$  % u^{n/(2n+2)}

Because this problem uses an infinite-dimensional feasible set $S$, computer solutions require \emph{discretization} and they are only approximate.  (\emph{Computers can only store finitely-many real numbers.  Storing arbitrary functions on an interval is not possible.})  The easiest way to discretize is to put a grid on the interval $I=[-100,100]$ and only consider functions which are piecewise-linear between the points of this grid.  In that case the derivative $u'(x)$ in the integral for $f[u]$ is computed by a difference quotient, the slope of a line segment between points.  Because this way of making the problem finite-dimensional is only an approximation, we want the grid to be as fine as practical given our tools, i.e.~the available optimization algorithms and computer resources.

\end{enumerate}

\end{document}

