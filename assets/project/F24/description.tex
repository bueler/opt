\documentclass[12pt]{amsart}
%prepared in AMSLaTeX, under LaTeX2e
\addtolength{\oddsidemargin}{-.5in}
\addtolength{\evensidemargin}{-.5in}
\addtolength{\topmargin}{-0.5in}
\addtolength{\textwidth}{1.1in}
\addtolength{\textheight}{1.0in}
\newcommand{\normalspacing}{\renewcommand{\baselinestretch}{1.05}
        \tiny\normalsize}

\newtheorem*{thm}{Theorem}
\newtheorem*{defn}{Definition}
\newtheorem*{example}{Example}
\newtheorem*{problem}{Problem}
\newtheorem*{remark}{Remark}

\usepackage{amssymb,fancyvrb,xspace}
\usepackage{palatino}

\usepackage[final]{graphicx}

\usepackage{tikz}
\usetikzlibrary{shapes,arrows,arrows.meta}


\usepackage[pdftex, colorlinks=true, plainpages=false, linkcolor=black, citecolor=red, urlcolor=red]{hyperref}

% macros
\newcommand{\ba}{\mathbf{a}}
\newcommand{\bb}{\mathbf{b}}
\newcommand{\bn}{\mathbf{n}}
\newcommand{\br}{\mathbf{r}}
\newcommand{\bu}{\mathbf{u}}
\newcommand{\bv}{\mathbf{v}}
\newcommand{\bx}{\mathbf{x}}
\newcommand{\by}{\mathbf{y}}

\newcommand{\bT}{\mathbf{T}}

\newcommand{\CC}{\mathbb{C}}
\newcommand{\Div}{\nabla\cdot}
\newcommand{\eps}{\epsilon}
\newcommand{\grad}{\nabla}
\newcommand{\ZZ}{\mathbb{Z}}
\newcommand{\ip}[2]{\ensuremath{\left<#1,#2\right>}}
\newcommand{\lam}{\lambda}
\newcommand{\lap}{\triangle}
\newcommand{\RR}{\mathbb{R}}

\newcommand{\prob}[1]{\bigskip\noindent\large\textbf{#1}.\,\normalsize }
\newcommand{\ppart}[1]{\textbf{(#1)}\,\, }
\newcommand{\epart}[1]{\medskip\noindent\textbf{(#1)}\,\, }

\newcommand{\pts}[1]{\scriptsize [#1 points] \normalsize}

\newcommand{\Matlab}{\textsc{Matlab}\xspace}
\newcommand{\Python}{\textsc{Python}\xspace}
\newcommand{\Julia}{\textsc{Julia}\xspace}


\begin{document}
\scriptsize \noindent Math 661 Optimization \, (Bueler) \hfill  28 October, 2024
\normalsize\bigskip
\normalspacing

\Large\centerline{\textbf{About your project}}
\normalsize

\bigskip\medskip
\thispagestyle{empty}
\normalspacing

\subsection*{Goal}  The goal of the Math 661 project is to focus on a topic of particular interest, and to become more familiar with specific optimization problems and algorithms than is possible with the brief coverage typical of the rest of the course.

\subsection*{Expectations}  All projects must include at least one specific optimization problem and at least one specific optimization algorithm.  You will implement at least one algorithm in a code you write in \Matlab/\Python/\Julia/etc.  Apply your code to at least one example problem.

However, your project may be application-driven (\emph{choose the problem(s) first}) or algorithm-driven (\emph{choose the algorithm(s) first}).  See the flowchart on page 3.

Both mathematical analysis and numerical computation are required.  Your analysis should use theory from the textbook\footnote{Griva, Nash, and Sofer, \emph{Linear and Nonlinear Optimization}, 2nd ed., SIAM Press 2009.} or from other references.  Analysis is important as it shows you have absorbed ideas from the course, and it compares algorithms.  Once your code is running you should provide some empirical (numerical experimentation) evidence regarding the error and/or performance of your algorithm(s).  Numerical evidence shows that you understood the algorithm well enough to implement it correctly.

The problem(s) you choose must be in the following form:
\begin{equation}
\min_{x\in \RR^n} f(x) \quad \text{subject to} \quad \begin{matrix}
                                                      g_i(x) = 0, & i \in \mathcal{E}, \\
                                                      g_i(x) \ge 0, & i \in \mathcal{I},
                                                      \end{matrix}  \label{genform}
\end{equation}
Of course you may replace $\min$ with $\max$.

Form \eqref{genform} describes a very large class of problems.  Yours must be well-enough understood to precisely identify the objective function $f(x)$ and a feasible set $S$ defined by finitely-many equality and inequality constraints $g_i(x)$.  It is o.k.~if there are no constraints, with $\mathcal{E}$ and $\mathcal{I}$ empty, but I may provide feedback on your Proposal that you consider a constrained form of your problem.

Your chosen problems must be finite-dimensional, but they may arise from an infinite-dimensional source.  If your project is algorithm-driven then you must identify which such problems are solved by your algorithms(s).


\subsection*{Due dates}  There are two due dates for the project:

\subsubsection*{Project Proposal:}  \textbf{Due Friday 8 November at the start of class.}

\smallskip
\noindent There are no format requirements for the Proposal, except that it must be \textbf{two pages or less}.  It should precisely say what problem(s) or algorithm(s) you will address.  If application-driven it should explain \emph{briefly} where the optimization problem(s) came from.  In any case it should briefly motivate your proposed choice(s) of algorithm(s).  Several quality references are expected; online references are o.k.  (However, many un-reviewed online documents are of low quality.)  Please make specific references to our textbook when that is appropriate.  Your Proposal should talk though what the complete Project will contain, to the degree possible.  Spending at least a few hours on thinking and research at this stage can be very effective, but I suggest that you spend at most 10 hours on the Proposal.

\subsubsection*{Project:}  \textbf{The completed project is due Friday 13 December at 5pm.}

\smallskip
\noindent It should have the format, specifically the indicated section headings, shown on page 4.  The total length \textbf{must be 20 pages or less}; I will not accept longer projects.  The total time spent on the whole project should be at most 25 hours.

The format expectations can be met by using the \LaTeX\xspace template posted online at \href{https://bueler.github.io/opt/projects.html}{\texttt{bueler.github.io/opt/projects.html}}, but this is certainly not required.


\subsection*{Choosing a topic}  I will help you choose problem(s) and algorithm(s) of appropriate difficulty, avoiding going too big.  The bigger the scope the easier it is to get lost in the application, or in difficulties with programming/debugging/analysis.  Your Proposal allows me to give good feedback on topic, variations, or a different analysis to consider.  I often suggest that you bite off less.

However, you may \textbf{not} choose a topic which is already adequately covered in lecture.  For example, neither the basic simplex method nor basic line search are good topics.  On the other hand, sparsity-respecting implementations of the simplex method would be a great choice (Chapter 7).  Comparing line search methods beyond backtracking, or quasi-Newton methods beyond BFGS, or trust-region methods, would all be good choices (Chapters 11 and 12).  Many constrained optimization algorithms go beyond the lecture, especially in Chapters 8, 10, and 16. 

Here are three approaches to choosing a topic if you don't already have one:

\subsubsection*{Approach 1: Inspiration from the Wikipedia page on mathematical optimization}

See the ``Major subfields,'' ``Computational \dots techniques'', and ``Applications'' sections.

   \centerline{\href{https://en.wikipedia.org/wiki/Mathematical_optimization}{\texttt{en.wikipedia.org/wiki/Mathematical\_optimization}}}

\subsubsection*{Approach 2: Investigate skipped material from the textbook}  Consider section(s) that you find interesting and which we did not cover.

\subsubsection*{Approach 3: A topic related to your thesis (if you have one)}  Please talk to your thesis advisor.  It is reasonable to ask ``are there optimization problems related to my expected thesis''?  There may be significant algorithms which arise in your field of interest, or problems like parameter fitting, inverse modeling, or optimal design.  There may be a paper to read about optimization in your field.  Please \textbf{do not} cover territory comparable to your whole thesis; instead extract a little part, or extend a little part, and do it carefully.  You must explain the context of your problem in your Proposal, but make this context brief and clear to an outsider.


\newpage
\subsection*{Structure of the project}  Here is a rough flow-chart.  It aligns with the section headings on the next page.

\bigskip

% Define block styles
\tikzstyle{decision} = [diamond, draw,
    text width=4.5em, text centered, node distance=3cm, inner sep=0pt]
\tikzstyle{block} = [rectangle, draw,
    text width=9em, text badly centered, rounded corners, minimum height=4em]
\tikzstyle{bigblock} = [rectangle, draw,
    text width=24em, text badly centered, rounded corners, minimum height=4em]
\tikzstyle{line} = [draw, -{Latex[length=3mm, width=2mm]}]

\begin{center}
\begin{tikzpicture}[node distance=2.4cm, auto, font=\small]
    % decide
    \node [decision] (decide) {your project is driven by};

    % algorithm sequence
    \node [block, below left of=decide, node distance=4cm] (introalg) {introduce algorithm(s)};
    \path [line] (decide.west) -- node [near start, left, yshift=4mm] {\textbf{algorithm}} (introalg.north);
    \node [block, below of=introalg] (algpseudo) {give pseudocode(s)};
    \path [line] (introalg) -- (algpseudo);
    \node [block, below of=algpseudo] (algexamples) {propose at least one example problem for testing};
    \path [line] (algpseudo) -- (algexamples);

    % application sequence
    \node [block, below right of=decide, node distance=4cm] (introapp) {introduce application(s)};
    \path [line] (decide.east) -- node [near start] {\textbf{application}} (introapp.north);
    \node [block, below of=introapp] (appexamples) {describe at least one example problem};
    \path [line] (introapp) -- (appexamples);
    \node [block, below of=appexamples] (appcompare) {describe at least one algorithm; give pseudocodes};
    \path [line] (appexamples) -- (appcompare);

    % merge and continue
    \node [block, below of=decide, node distance=10.5cm] (implement) {implement algorithms in \Matlab, \Python, \Julia, \dots};
    \path [line] (algexamples) -- (implement);
    \path [line] (appcompare) -- (implement);
    \node [block, below of=implement] (results) {demonstrate runs on example(s); show results};
    \path [line] (implement) -- (results);
    \node [bigblock, below of=results] (analysis) {analysis: \begin{itemize}
       \item convergence: e.g.~state theorems; compare rates
       \item performance: e.g.~count operations; show timing
       \end{itemize}     };
    \path [line] (results) -- (analysis);
    \node [block, below of=analysis] (conclude) {what you would do next? conclude};
    \path [line] (analysis) -- (conclude);
\end{tikzpicture}
\end{center}

\clearpage
\newpage

\includegraphics[height=\textheight]{blank.pdf}
\end{document}
