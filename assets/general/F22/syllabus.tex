\documentclass[12pt]{article}

% Layout.
\usepackage[top=1.2in, bottom=0.9in, left=1in, right=1in, headheight=1in, headsep=6pt]{geometry}

% Fonts.
\usepackage{mathptmx}
\usepackage[scaled=1.0]{helvet}
\renewcommand{\emph}[1]{\textsf{\textbf{#1}}}

% Misc packages.
\usepackage{amsmath,amssymb,latexsym}
\usepackage{graphicx,hyperref}
\usepackage{array}
\usepackage{xcolor}
\usepackage{multicol}
\usepackage{tabularx,colortbl}
\usepackage{enumitem}

\hypersetup{
    colorlinks=true,
    linkcolor=blue,
    filecolor=magenta,      
    urlcolor=blue,
    pdftitle={Syllabus for MATH F661 Fall 2022 (Bueler)},
    pdfpagemode=FullScreen,
    }

\def\mailto#1{\href{mailto:#1}{#1}}

% Paragraph spacing
\parindent 0pt
\parskip 6pt plus 1pt
\def\tableindent{\hskip 0.5 in}
\def\ts{\hskip 1.5 em}

\usepackage{fancyhdr}
\pagestyle{fancy} 
%\chead{\large\sf\textbf{}}
\lhead{\large\sf\textbf{Syllabus} \normalsize (revised)}
\rhead{\large\sf\textbf{MATH F661 Optimization, Fall 2022}}
  
\newcommand{\localhead}[1]{\par\smallskip\textbf{#1} \smallskip\nobreak\\}%
\def\heading#1{\localhead{\large\emph{#1}}}
\def\subheading#1{\localhead{\emph{#1}}}

\newenvironment{clist}%
{\bgroup\parskip 0pt\begin{list}{$\bullet$}{\partopsep 4pt\topsep 0pt\itemsep -2pt}}%
{\end{list}\egroup}%

\begin{document}

\strut\par\vskip-12pt
\heading{Essential Information}

\vskip -12pt
\strut\hbox to \hsize{\tableindent\vtop{\halign{#\hfill\ts&#\hfil\cr
{\emph{Instructor}} & Ed Bueler \quad \href{mailto:elbueler@alaska.edu}{\texttt{elbueler\@@alaska.edu}} \cr
\strut & \cr
{\emph{Class meeting}} & MWF 3:30--4:30 pm, Chapman 206 \cr
\strut & \cr
{\emph{CRNs}} & in-person 76523, online 76522\cr
\strut & \cr
{\emph{Public website}} & \href{https://bueler.github.io/opt/}{\texttt{bueler.github.io/opt/}}\cr
\strut & \cr
{\emph{Canvas website}} & \href{https://canvas.alaska.edu/courses/9948}{\texttt{canvas.alaska.edu/courses/9948}} \cr
\strut & \cr
\emph{Prerequisite} & Knowledge of calculus, linear algebra and computer programming.\cr
\strut & \cr
{\emph{Required text}} & Griva, Nash, \& Sofer, \emph{Linear and Nonlinear Optimization},\cr
  & 2nd ed., SIAM Press 2009 \cr
}
\hfil}}

\heading{Description and Topics}
Optimization is essential mathematical technology for science, engineering, statistics, machine learning, and finance.  This graduate-level introduction focusses on ideas, algorithms, and applications, but it uses mathematical rigor (theorems and proofs) when appropriate.  The course will combine lectures, in-class activities, homework assignments which include both writing and programming, a project reflecting the student's particular interests, and in-class exams.

We cover these topics:
\begin{itemize}[nosep]
\item Continuous optimization, both nonlinear and linear.
\item Iterative methods for unconstrained problems:
    \begin{itemize}[nosep]
    \item gradient descent
    \item Newton and quasi-Newton methods
    \item conjugate gradients
    \end{itemize}
\item Methods for constrained problems:
    \begin{itemize}[nosep]
    \item simplex method for linear programming
    \item interior point methods
    \end{itemize}
\item Line search and trust region methods.
\item Constrained problems and the Karush-Kuhn-Tucker conditions.
\item Linear algebra related to the above topics.
\item Convergence theorems for some methods.
\item Examples from applications.
\item Practical work with a scientific computing language.
\end{itemize}
(``Linear programming'' means optimization of a linear function subject to linear constraints.)

\heading{Course Goals and Student Learning Outcomes}
The goal of this applied mathematics course is to be able to understand optimization problems as they arise in applied contexts.  At the end of the course you should be able to select algorithms and apply optimization software based on an understanding of theory and standard examples.  Understanding of concepts should suffice for the student to assess claims about optimization software performance.  Increased student competence with general scientific computing, using languages like Matlab or Python, is also a goal.


\heading{Schedule and Online Materials}
The \href{https://bueler.github.io/opt/}{public course website} includes a \href{https://bueler.github.io/opt/assets/general/schedule.pdf}{schedule} listing the textbook sections to be covered during each lecture, the due date of each homework Assignment, and the dates for the Midterm and Final Exams.  Please consult this schedule frequently; it is subject to change and will be kept up to date.

Most course materials (syllabus, schedule, homework Assignments, code examples, project description, etc.) will be posted on the \href{https://bueler.github.io/opt/}{public course webpage}.  Some course materials (student grades, homework and exam solutions, etc.) will go on the \href{https://canvas.alaska.edu/courses/9948}{Canvas site}.

The zoom link for getting the lecture online is also on the \href{https://canvas.alaska.edu/courses/9948}{Canvas site}.

\heading{Office Hours and Communication}
My Office Hours are shown online at \href{http://bueler.github.io/OffHrs.htm}{\texttt{bueler.github.io/OffHrs.htm}}; I hold office hours in Chapman 306C.  Students can also schedule meetings with me outside of regular office hours.  I will use Canvas to send announcements.  If I need to contact you outside of class times, I'll try to email via Canvas, so please set your email address in Canvas to one that you check regularly!


\heading{Evaluation and Grades}
(Revised 29 November.)  Grades are determined as follows.

\begin{tabular}{|c|c|c|}
\hline
Homework & nearly weekly & 50\% \\
\hline
Project Part I & due Friday 11 November & 5\%  \\
\hline
Project Part II & due Monday 12 December & 15\%  \\
\hline
Midterm Exam & in-class Friday 28 October & 20\%  \\
\hline
Final Exam & \, in-class \emph{Friday 16 December}, 3:15--5:15pm \, & 10\% \\
\hline
total & & 100\% \, \\
\hline
\end{tabular}

Scores for specific assignments/projects/exams may be adjusted based on the actual difficulty of the work and/or on average class performance.  Any such adjustments will be applied to all students equally.  The scores of the various parts will be summed and the final course grade will be assigned as follows.

\begin{tabular}{llll}
A  & 93--100\%& C  & 68--75\%  \\
A- & 90--92\% & C- & not given \\
B+ & 87--89\% & D+ & 65--67\%  \\
B  & 82--86\% & D  & 60--65\%  \\
B- & 79--81\% & D- & 57--59\%  \\
C+ & 76--78\% & F  & $\le$ 56\%
\end{tabular}

These ranges are a guarantee and a lower bound.  I reserve the right to increase your grade above these ranges based on the actual difficulty of the work and/or on average class performance.  Any such increases will preserve grade ordering by weighted total score.


\clearpage\newpage
\strut\vspace{-10pt}

\heading{Homework}
The homework consists of by-hand computations, design and analysis of numerical algorithms, computer implementation of those algorithms, by-hand and computer visualization of functions and sets, rigorously-justified examples and counter-examples, and some proofs.

Examples in lecture and exercises on the homework will use Matlab/Octave, both as a supercalculator and as a programming language, and help will be given to learn Matlab/Octave.  While Matlab/Octave is well-suited to implementing optimization algorithms, other languages are accepted for all student work.  However, only Matlab/Octave is fully instructor-supported, so, for example, codes on homework solutions will only be in Matlab.  See the separate document \textsl{Programming languages compared} (\href{https://bueler.github.io/compareMOP.pdf}{\texttt{bueler.github.io/compareMOP.pdf}}) for other recommended scientific computing languages.

Homework assignments and their due dates will regularly be posted at the \href{https://bueler.github.io/opt/}{\texttt{public website}}.  The site also has a daily schedule of topics.  The schedule will be updated on an ongoing basis to reflect which topics were actually covered each day, so it is subject to change.  The public website will also have links to a growing list of short Matlab codes; this is a good resource for coding examples.

Late Assignments will not be accepted.  If you have unavoidable circumstances which do not allow you to turn in an Assignment on time then please contact me (\href{mailto:elbueler@alaska.edu}{\texttt{elbueler\@@alaska.edu}}) in advance.

Problems very similar to, or shortened versions of, Homework problems will appear on the in-class Exams.

\heading{Project}
The project is in two parts, with the first part due midsemester and the second due just before final exams (dates above).  The topic will mostly be up to you, but I will make suggestions, and I reserve veto power on choice of topics.  The project must include both theory and practical computation.  A detailed handout will appear on Monday 31 October, outlining how you might choose a project, and what are the expectations.

\heading{Exams}
There will be one in-class Midterm Exam covering mostly basic concepts and definitions. The in-class Final Exam will have similar problems from the whole semester, but weighted toward the latter half of the course.

A make-up Midterm will be given only for documented extenuating circumstances, at my discretion.  Department policy (below) does not allow me to move the time of the Final Exam.


\heading{Rules and Policies}
\vskip -20pt

\subheading{Incomplete Grade} 
Incomplete (I) will only be given in
  DMS courses in cases where
  the student has completed the majority (normally all but the last
  three weeks) of a course with a grade of C or better, but for
  personal reasons beyond his/her control has been unable to complete
  the course during the regular term. Negligence or indifference are
  not acceptable reasons for granting an incomplete grade.

\clearpage\newpage
\strut\vspace{-10pt}
\subheading{Late Withdrawals} 
A withdrawal after the deadline from a DMS course will
  normally be granted only in cases where the student is performing
  satisfactorily (i.e., C or better) in a course, but has exceptional
  reasons, beyond his/her control, for being unable to complete the
  course.  These exceptional reasons should be detailed in writing to
  the instructor, Department Chair and the Dean.

\subheading{No Early Final Examinations}
Final examinations for DMS courses shall not be held earlier than the date and time published in the official term schedule.  Normally, a student will not be allowed to take a final exam early.  Exceptions can be made by individual instructors, but should only be allowed in exceptional circumstances and in a manner which doesn't endanger the security of the exam.

\subheading{Academic Dishonesty}
Academic dishonesty, including cheating and plagiarism, will not be tolerated.  It is a violation of the Student Code of Conduct and will be punished according to UAF procedures.

\subheading{Student protections and service statement}
Every qualified student is welcome in my classroom.  As needed, I am happy to work with you, Disability Services, Veterans' Services, Rural Student Services, and so on, to find reasonable accommodations.  Students at this University are protected against sexual harassment and discrimination (Title IX), and minors have additional protections.  For more information on your rights as a student and the resources available to you to resolve problems, please go the following site: \href{https://www.uaf.edu/handbook/}{\texttt{www.uaf.edu/handbook}}.

\hfill  \scriptsize [syllabus version: \today] \normalsize

\end{document}
